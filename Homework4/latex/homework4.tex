
\documentclass[10pt]{article}
\usepackage{graphicx}
\usepackage{amsmath}
\usepackage{rotating}
\usepackage{lscape}
\usepackage[margin=1.0in]{geometry}
\usepackage{float}
\usepackage{wrapfig}
\usepackage{subfigure}
\usepackage{circuitikz}
\usepackage{caption}
\usepackage{verbatim}

\begin{document}


\section{Question 1}
\subsection{Explanation}
The program uses simpsons integration and bisection root finding to evaluate the limits of the integral. By creating an equation where the intgral minus a half is equal to zero we use the bisection rule to find which limits bring that integral value close to a half. We keep doing this until we have attained the five digits in precision of a and I.\subsection{Code:}
\verbatiminput{homework1.c}
\subsection{Output:}
\verbatiminput{out1.txt}
From this output we determine $a = 0.6745201$ and that is accurate to five digits of precision due to the fact that both the values in a and I after the last iteration changed by an order less than five digits.\section{Question 2}
\subsection{Explanation}
For the given anhamornic oscilitor, this program integrates over the ode with both a fourth order sympletic algorighm and fourth order runge-kutta. At each iteration the program outputs the deviation of the energy produce from each algorighm from the orginal energy. At the same time keeps track of  the maximum deviation.\subsection{Code:}
\verbatiminput{homework2.c}
\subsection{Output:}
\verbatiminput{out2.txt}
It goes to show that at low periods the runge kutta algorighm maybe more precision since its energy deviates, but at higher time periods RK4 deviation growths while PERFL seems to ocsiliate around the intial energy.\section{Question 3}
\subsection{Explanation}
\subsection{Code:}
\verbatiminput{homework3.c}
\subsection{Output:}
\verbatiminput{out3.txt}
\section{Question 4}
\subsection{Explanation}
This program uses an Rk4 method, decoupling the 2nd order schroedinger equation into the first derivative of the wave amplitude and the wave amplitude we step foward from 0 to 1. We do this for varying step sizes.\subsection{Code:}
\verbatiminput{homework4.c}
\subsection{Output:}
\verbatiminput{out4.txt}
We can see that as we make the step size smaller the value of  $\psi(1)$ is approaching zero. Which shows that smaller step sizes make the Rk4 algorighm more accurate. This is due to the fact that the algorighm advances by evaulating derivates at that step size. \section{Question 5}
\subsection{Explanation}
This program will create an an array of random numbers and then call heapsort.c . Heapsort first builds a heap by starting at $\frac{N}{2}$ and checking the heap property in this case that element i is larger that the element at either 2i or 2i +1(known as children). By doing this we have the largest element at the top of the heap. We then enter the second phase where we swap the last element with the top element and then check if it larger than the children if not we swap those elements recursively. After the elements are sorted we run a simple bisection algorighm (or binary search) where we start from the end and begining look at the middle and depending if the element we are looking for is less or more than than he middle we recusively search into that section.\subsection{Code:}
\verbatiminput{homework5.c}
\subsection{Output:}
\verbatiminput{out5.txt}
When running heap sort on $10^6$ elements versus $10^7$ there is only a multiplitive factor of $11.7$ between the number of operation needed to be done. This is measured by the amount of times we must check the parent to the children.\section{Question 6}
\subsection{Explanation}
This program reads in x,y, and error values from a file. By calculating the values for the matrix U represented by an array. After that we $a_0$ the intercept and $a_1$ the slope and their respective errors are simply computed from this array. Finally the function chi sq() calculates the $\chi^2$ value. Afterwards for part c the program calculate the Quality by computing $1 - \frac{1}{\Gamma(\frac{M}{2})} \int_0^{\frac{\chi^2}{2}} y^{\frac{M}{2}-1}e^{-y} dy$ \subsection{Code:}
\verbatiminput{homework6.c}
\subsection{Output:}
\verbatiminput{out6.txt}
We take to be the number of degree of freedom to be the number of x variables minus the two constraints from $a_0$, $a_1$. We find the quality factor to be .32 which proves it to be a good fit.\subsection{PERLF.c}
\verbatiminput{PERFL.c}
\end{document}
